\documentclass{article}
\usepackage[margin=1in]{geometry}
\title{For students who are struggling}
\author{Madhusudan Singh\thanks{Functional Materials \& Devices Laboratory}}
\date{June 2019}
\begin{document}

\maketitle

I am assuming that you have already set yourself up to work in the FMDL way with the README file in the repo. This document contains some helpful hints for students who might be struggling with putting together a draft. This is understandable - this may be the first time you have put together any serious document.

Publication, and scientific communication is central to the business of our corner of academics. You are likely reading this document as a first-year grad (MS or Ph. D.) student working in the general area of devices \& Materials. I am not going to go into the philosophy of writing an article, vital as it is, but will focus on operational choices that will get you up to speed as soon as possible.

Before we get started with the mechanics, if you are the kind of person who skims through instructions because you think that you are too good for clearly written instructions, then please stop reading this, clear out your desk if you have one, and find another advisor. If you can't or won't read, you do not have a future in research.

%Besides I have very little patience with people who waste my time because they cannot be bothered to read and follow instructions in matters such as these. Agreeing to advise you is a time-investment on my part. Putting together these instructions took time, If I find that your work ethic involves any variant of unprofessional \textit{dekh lenge chalta hai} nonsense, then spare us both the aggravation, and just leave. There are plenty of excellent and serious students who want to join, and make something of their lives, and you are just taking up space that does not belong to you.

If you are joining my lab as an engineering or sciences student without the benefit of an ICSE or comparable high school education, chances are that you have issues in clearly, conicsely and forcefully expressing yourself verbally and in the written form. This has possibly affected your confidence and made you think less of yourself. While these sentiments are understandable, they are without basis in reality. Lack of facility with expressing yourself does not make you a worse scientist, only an ineffective one. This handicap means that you may be great in the lab, but no one but your advisor (if that) will ever know that about you.

This is a (slowly) fixable problem. Please buy (or otherwise find) a copy of The Complete Sherlock Holmes: 2 Boxes Sets by Sir Arthur Conan Doyle (ISBN: 978-0553328257) from your favourite bookseller. Read this book for fun everyday, with no page targets. Don't let it interfere with your work, but use it for relaxation. It will take several months to undo the damage you have had to undergo in high school, but if you develop a habit of reading \textit{non-scientific} literature in your spare time, your command over English will get better.

\section{How to produce a manuscript suitable for editing and submission in the shortest possible time}

If you follow these instructions below, you will be able to \textbf{write} a defensible first draft of a manuscript in \textbf{six} hours. Of course, this assumes that your initial experiments are done, your figures are ready to go. Use the manuscript draft template in this repo - that has some more specific instructions.



\section{The soul of the paper: figures and arguments}


The root of the problem is sometimes some missing common sense, lack of seriousness, not being systematic.

Authoring of manuscripts. The editing process for a draft involves text changes, bibliography additions/deletions/changes, etc.

- The standard workflow for writing manuscripts is LaTeX/Zotero/Grammarly/TurnitIn/github. At least one complete cycle of this workflow should be executed before you send in the first draft. Your advisors are not your copy-editors, or plagiarism detectors, etc.

%- The standard section order as taught in ELV734/ELL834, is I-M-R-A-D-C-A. Introduction. Methods. Results & Discussion. Conclusions. Acknowledgments. This is so, regardless of the journal you are targeting. Even if it is a Letters journal without explicit sections.

- When you send me a draft, and it is not in the standard  workflow (such as rare cases where you are using Word), the bibliographic fields have to be editable. Not plain text. The same applies to Figure and Table cross-references. If you convert the fields to plain text, you are basically telling me that the number and ordering of the cross/references is sacrosanct, and will not be changed in editing. The slightest exertion of thinking would reveal that this is never the case, right until submission of the manuscript. For Word files, two files must always exist in your github repository - a) one draft with field codes active, b) one draft with field codes removed. The second file is what we submit. When stuff comes back for revisions, we edit a), and generate b).

- When you send me a draft that is written in broken glass form (sentences make little to no sense, methods are not described in process chronological order, stuff from different sections is mixed up, thoughts have no transition from one to the next, etc.), you are telling me that I do not have to take the manuscript seriously. I cannot possibly be more serious about your Ph. D. progress than you are. All you will accomplish with broken glass manuscripts is wasted time as I will just bounce it back to you, and develop a lower opinion of your level of commitment.

- Unsupported contentions. When matters of known facts are unreferenced, you are effectively claiming primacy over knowledge developed by others. This is one rung below plagiarism, and faces consequences that are nearly as severe. If you want a reviewer to give you really nasty report, by all means, continue with this unethical behaviour, and mess up your progress. No, supporting references are not sweet cherries that are distributed at the end. We remove unnecessary references at the end of the editing process, not add. Sentences in your draft and supporting references have to be added at the same time.

Authoring of presentations / posters.

- The attributions for figures (and less commonly, tables) take the following forms:
	a) (own work): everyone who helped you, as “Courtesy: X. Y. ZZZZ” in small red text. In case or NRF and CRF, it is “Courtesy: NRF/CRF”. If the measurement was done by a group colleague, it is “Courtesy: Mr. XYZ”, etc.
	b) (others’ work): citation in shortened form “Kim, et al, Appl. Phys. Lett., 2018” in small and red text.
	c) (own work, previously published), same as b, but with all last names, with your own name underlined.
	d) (own work, previously unpublished), same as c, but with current status of the publication listed (“In press”, “submitted”, etc.).
	e) (other’s work, uncited, discussion - this will be very rare), same as b) but with the journal name replaced with “Private communication”.

- The attributions when coupled with figures must appear in a group with the figure. That ensures that when you use builds, the attribution and the figure appear together at the same time. At no time should an unattributed piece of data be present on your slides during presentation. No, they are not going to wait for the next click.

\end{document}
