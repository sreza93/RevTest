% Before you get started, replace "WXYZ-N" in the name of this file with the following string: where WXYZ are a four letter code for the journal (sometimes extractable from the reference code they assigned you after original submission - you can use our own (see below)). Examples:
% IEEE Sensors J: SENJ
% IEEE Sensors Letters: SENL
% physica status solid (A): PSSA
% MRS Advances: MRSA
% Applied Physics Letters: APLT
% J. Appl. Phys.: JAPS
% RSC Green Chemistry: RGRC
% Adv. Func. Mater: ADFM
% Adv. Mater.: ADMA
% MRS Communications: MRSC
% Phys. Rev. Lett.: PRLT
% Phys. Rev. B: PRVB

% N is the number of the review cycle you are in. If this is the first review report that we are responding to, N=1.

% So, for a review response to Advanced Functional Materials, where this is our second response to the reviewers, the name of the file should be ReviewerResponseADFM-2.tex

% We could have chosen to keep things simple and rely on GitHub to store old versions. The problem with that idea is that sometimes you need to cite/attach earlier responses in communications with journals, and it is more sensible to have an unmodified document on hand.


\documentclass[12pt]{article}
%%%%%%%%%%%%%%%%%%%%%%%%%%%%%%%%%%%%%%%
% Required packages
\usepackage[margin=1in,footskip=0.25in]{geometry}
\usepackage{longtable}
\usepackage{graphicx}
\usepackage{setspace}
\usepackage{textcomp}
\usepackage{gensymb}
\usepackage{physics}
\doublespacing
\usepackage{subfigure}
\usepackage{float}
%\usepackage{caption}
\usepackage[version=4]{mhchem}
\usepackage{amsmath}
\usepackage{authblk}
\usepackage{romannum}
\usepackage{caption} % For Table of Contents entry

\usepackage[backend=biber, isbn=false, url=false, uniquename=init, terseinits=true, hyperref=true, doi=false, style=numeric, defernumbers=false, autocite=plain, sorting=none]{biblatex}
\addbibresource{publishing.bib} % Replace this with the name of your .bib file.
\usepackage[separate-uncertainty = true,multi-part-units=single]{siunitx}
\DeclareSIUnit{\Molar}{M}
\DeclareSIPrefix\micro{\text{\textmu}}{-3}
\DeclareSIUnit\angstrom{\text {Å}}
\DeclareSIUnit\gauss{G}

\setlength{\marginparwidth}{2cm}\usepackage[colorinlistoftodos]{todonotes}
\newcommand{\addcitation}[1]{\todo[inline,color=green!40]{\textbf{Add citation: }#1}}
\newcommand{\adddiscussion}[1]{\todo[inline,color=blue!40!green]{\textbf{Add discussion: }#1}}
\newcommand{\explain}[1]{\todo[inline,color=red!40!yellow]{\textbf{Explain: }#1}}
\newcommand{\reviewthis}[1]{\todo[color=yellow,inline]{\textbf{Review: }#1}}
\newcommand{\response}[3]{\todo[color=yellow!90!red,inline]{\textbf{Response to Reviewer #1 (comment #2): }#3}}
\newcommand{\rework}[1]{\todo[color=orange,inline]{\textbf{Rework: }#1}}
\newcommand{\incorrect}[1]{\todo[inline,color=red!70]{\textbf{Incorrect: }#1}}
\newcommand{\rephrase}[1]{\todo[inline,color=blue!40]{\textbf{Rephrase: }#1}}
\newcommand{\warning}[1]{\todo[color=red,inline]{\textbf{Warning: }#1}}
\newcommand{\modifyfigure}[1]{\todo[inline,color=yellow!80!blue]{\textbf{Modify figure: }#1}}
\newcommand{\pleaseconfirm}[1]{\todo[inline,color=red!80!blue]{\textbf{Please confirm: }#1}}
\newcommand{\query}[1]{\todo[color=gray,inline]{\textbf{Query: }#1}}
\newcommand{\missingacknowledgment}[1]{\todo[inline,color=red!80]{\textbf{Missing acknowledgment: }#1}}
\newcommand{\restructuredocument}[1]{\todo[inline,color=blue!50]{\textbf{Restructure document: }#1}} %The choice of color is very harsh on the eyes. Please keep it low.
\newcommand{\nice}[1]{\todo[color=green,inline]{\textbf{Nice: }#1}}
\newcommand{\done}[1]{\todo[inline,color=blue!10]{\textbf{Done: }#1}}
\newcommand{\reply}[1]{\todo[color=yellow!90!red,inline]{\textbf{Reply: }#1}}
 % To be commented out prior to submission.

% \renewcommand{\finentrypunct}{
%   \addperiod\space
%   (Cited \arabic{citecounter}~time\ifnumequal{\value{citecounter}}{1}{}{s})%
%  }
\usepackage{hyperref}
\graphicspath{{./figures/}}

\title{Response to reviewer comments \\
  \large \textsf{Title of your paper} (submission reference code from journal - typically a single alphanumeric string)}

\author[1]{\small First Author} % ORCID ID of the first author:
\author[2]{\small Second Author}  % ORCID ID of the second author:
\author[1]{\small Third Author}  % ORCID ID of the third author:
\author[1]{\small Postdoc Scholar}  % ORCID ID of the postdoc involved:
\author[2]{\small Collaborating PI}  % ORCID ID of the collaborating PI:
\author[1]{\small Madhusudan Singh\thanks{msingh@ee.iitd.ac.in}}  % ORCID ID: 0000-0001-5430-1405
\affil[1]{\footnotesize Functional Materials \& Devices Laboratory, Department of Electrical Engineering, IIT Delhi, New Delhi, NCT, India, 110 016.} % Our lab - will remain untouched.
\affil[2]{\footnotesize Department of Chemistry, IIT Delhi, New Delhi, NCT, India, 110 016.} % Change/comment out as needed

\date{}
\begin{document}

\maketitle

We would like to thank the editor for arranging a careful review of this manuscript. We have gone through the reviewer comments in detail, and as requested, have carried out estimates of the obtained quantities after including corrections pointed out by the reviewers. We are thankful to the reviewers for their incisive and helpful comments.

Below, we list the changes made in response to reviewer comments and requests. The changes made have been incorporated in the manuscript as per requirements. The related figures have also been updated in the manuscript, if needed. For your convenience, a marked copy of the manuscript also submitted with changes in text and equations marked (red=deletion, blue=addition).

\textbf{Note to student}: No actual changes in the manuscript are supposed to be included below. Only their locations, and the reasoning/response are located in this document.

\textbf{Note to student}: No changes are marked up in the original document. The deletion (red) and addition (blue) markup is generated by my shell script that compares two checkpoints: a) last submission, b) pre-final draft and generates the difference tex file and pdf document automatically using git and latexdiff. Do not waste your time on marking up changes in the main document.

\begin{enumerate}
\item \textbf{Reviewer 1} % You can add as many items as needed below for this reviewer. Break up larger paragraphs into separate items which pertain to separate points. All text by the reviewer must be quoted in textit's, without \textbf{any} modifications. We are quoting the reviewer - it must be verbatim.
  \begin{enumerate}
  \item \textit{Comment 1 by reviewer 1.}
    
    Your response goes here. As shown in previous work\cite{goossens:1994}, the quantity A is expected to increase when parameter B is reduced. Our results are consistent with that. However, the reviewer's criticism that a different starting concentration could have been used to improve the S/N ratio is very helpful. Accordingly, we have repeated our measurements (results are in Fig.~4(c) of the main manuscript). The resulting changes have been summarized in Table~(\ref{tbl:discrepancy}).

    \begin{table}[!ht]
      \centering
      \caption{Summary of discrepancy from earlier submitted results. This table should not be in the main manuscript.}
      \label{tbl:discrepancy}
      \resizebox{10cm}{!}{%
        \begin{tabular}{ccc}
          \hline
          Frequency (\SI{}{\kilo\hertz}) & Original submission~$A$(\SI{}{\pico\meter\per\volt}) & Revised $A$(\SI{}{\pico\meter\per\volt}) \\
          \hline
          100 & 15.502 & 15.518  \\
          300 & 15.589 & 15.606  \\
          500 & 15.361 & 15.381  \\
          700 & 15.532 & 15.551  \\
          900 & 15.496 & 15.512  \\
          1000 & 15.442 & 15.459 \\
          \hline
        \end{tabular}}
    \end{table}
    
    Changes marked in the manuscript:
    
    \begin{itemize}
    \item On page 3, column 1, line 15: we have ...
    \item On page 4, column 2, line 2: we have ...
    \item Figure 4(c) has been updated with the new results.
    \item References [36] and [48] have been added to the manuscript.
    \end{itemize}
    
  \item \textit{Comment 2 by reviewer 1.}
    
    Your response goes here.
    
    Changes marked in the manuscript:
    
    \begin{itemize}
    \item On page 1, column 2, line 6: we have ...
    \item On page 5, column 1, line 20: we have ...
    \item Figure 2(b) has been updated with the new results.
    \item References [11] and [42] have been added to the manuscript.
    \end{itemize}
  \end{enumerate}
  
\item \textbf{Reviewer 2} % You can add as many items as needed below for this reviewer. Break up larger paragraphs into separate items which pertain to separate points. All text by the reviewer must be quoted in textit's, without \textbf{any} modifications. We are quoting the reviewer - it must be verbatim.
  \begin{enumerate}
  \item \textit{Comment 1 by reviewer 2.}
    
    Your response goes here.

    \begin{figure}
      \centering
      \includegraphics[width=88mm,keepaspectratio]{example-image-a}
      \caption{Use clear images, preferably in .eps form for anything other than scan data, and (b) .png form for scan data. The width of the figure is chosen to be 88mm since that is the typical width of a journal column in two-column format. This figure is meant \textbf{purely} for the reviewer's benefit - it is not included in the main manuscript. The need for use of this figure will be apparent from the nature of the comments.}
      \label{fig:cv}
    \end{figure}
    
    Changes marked in the manuscript:
    
    \begin{itemize}
    \item On page 3, column 1, line 15: we have ...
    \item On page 4, column 2, line 2: we have ...
    \item Figure 4(c) has been updated with the new results.
    \item References [36] and [48] have been added to the manuscript.
    \end{itemize}
    
  \item \textit{Comment 2 by reviewer 2.}
    
    Your response goes here.
    
    Changes marked in the manuscript:
    
    \begin{itemize}
    \item On page 1, column 2, line 6: we have ...
    \item On page 5, column 1, line 20: we have ...
    \item Figure 2(b) has been updated with the new results.
    \item References [11] and [42] have been added to the manuscript.
    \end{itemize}
  \end{enumerate}
  
\item \textbf{Reviewer 3} % You can add as many items as needed below for this reviewer. Break up larger paragraphs into separate items which pertain to separate points. All text by the reviewer must be quoted in textit's, without \textbf{any} modifications. We are quoting the reviewer - it must be verbatim.
  \begin{enumerate}
  \item \textit{Comment 1 by reviewer 3.}
    
    Your response goes here.

    Changes marked in the manuscript:

    \begin{itemize}
    \item On page 3, column 1, line 15: we have ...
    \item On page 4, column 2, line 2: we have ...
    \item Figure 4(c) has been updated with the new results.
    \item References [36] and [48] have been added to the manuscript.
    \end{itemize}

  \item \textit{Comment 2 by reviewer 3.}

    Your response goes here.

    Changes marked in the manuscript:

    \begin{itemize}
    \item On page 1, column 2, line 6: we have ...
    \item On page 5, column 1, line 20: we have ...
    \item Figure 2(b) has been updated with the new results.
    \item References [11] and [42] have been added to the manuscript.
    \end{itemize}
  \end{enumerate}
  
\end{enumerate}

We hope that the changes made in response to specific reviewer comments address their important concerns. We look forward to your further decision on this manuscript in this regard.

\textbf{Note to student}: Generally, once you have gone through 1-2 major revision / reject and resubmit decisions, the tone of the decision letter from the editor will make it clear whether they are hoping to accept the paper, or if they want production data. Some journals don't say anything in the decision letter, but during submission process, you have to provide production data for the publication process. The details of this vary from journal to journal. However, the source tex file (with the printbibliography replaced with the contents of the .bbl file), and high resolution figures are usually needed.  After preparing your reviewer response, it would be wise to provide these files under \verb|DOCNAME/acceptance_files| so that I do not have to harangue you for those files after we are ready to submit everything. There is a README file in that folder that tells you what to put where.

\printbibliography
\end{document}